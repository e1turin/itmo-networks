%%%%%%%%%%%%%%%%%%%%%%%%%%%%%%%%% LAB-5 %%%%%%%%%%%%%%%%%%%%%%%%%%%%%%%%%%
%>>>>>>>>>>>>>>>>>>>>>>>>>> ПЕРЕМЕННЫЕ >>>>>>>>>>>>>>>>>>>>>>>>>>>>>>>>>>>
%>>>>> Информация о кафедре
%\newcommand{\year}{2021 г.}  % Год устанавливается автоматически
\newcommand{\city}{Санкт-Петербург}  %  Футер, нижний колонтитул на титульном листе
\newcommand{\university}{Национальный исследовательский университет ИТМО}  % первая строка
\newcommand{\department}{Факультет программной инженерии и компьютерной техники}  % Вторая строка
\newcommand{\major}{Направление программная инженерия}  % Треьтя строка
% Пусть будет. Проще закоментить лишнее.
\newcommand{\education}{Образовательная программа системное и прикладное программное обеспечение}  % четвертая строка
\newcommand{\specialization}{Специализация системное программное обеспечение}  % пятая строка

%<<<<< Информация о кафедре

%>>>>> Назание работы
\newcommand{\reporttype}{ОТЧЕТ ПО ДОМАШНЕЙ РАБОТЕ} % тип работы, (главный заголовок титульного листа)
\newcommand{\lab}{Лабораторная работа}          % вид работы
\newcommand{\labnumber}{№ 3}                    % порядковый номер работы
\newcommand{\subject}{Компьютерные сети}         % учебный предмет
\newcommand{\labtheme}{Моделирование компьютерных сетей в среде NetEmul: Компьютерные сети с маршрутизаторами}            % Тема лабораторной работы

\newcommand{\student}{Тюрин Иван Николаевич}    % определение ФИО студента
\newcommand{\studygroup}{P33102}                 % определение учебной группы 
\newcommand{\teacher}{% принимающий
    Авксентьева Е. Ю.,\\[1mm]% ФИО лектора
    Алиев Т. И.% ФИО практика
}

%>>>>>>>>>>>>>>>>>>>>>> ПРЕАМБУЛА >>>>>>>>>>>>>>>>>>>>>>>>>
\include{preamble}
%<<<<<<<<<<<<<<<<<<<<<< ПРЕАМБУЛА <<<<<<<<<<<<<<<<<<<<<<<<<



%%%%%%%%%%%%%%%%%%% СОДЕРЖИМОЕ ОТЧЕТА %%%%%%%%%%%%%%%%%%%%%
%>>>>>>>>>>>>>>> ''''''''''''''''''''''' >>>>>>>>>>>>>>>>>>
\begin{document}


%>>>>>>>>>>>>>>>> ОПРЕДЕЛЕНИЕ НАЗВАНИЙ >>>>>>>>>>>>>>>>>>>>
% Переоформление некоторых стандартных названий
%\renewcommand{\chaptername}{Лабораторная работа}
\renewcommand{\chaptername}{\lab\ \labnumber} % переименование глав
\def\contentsname{Содержание} % переименование оглавления
%<<<<<<<<<<<<<<<< ОПРЕДЕЛЕНИЕ НАЗВАНИЙ <<<<<<<<<<<<<<<<<<<<
% \setlength{\itemsep}{0pt} % установка расстояния между строчками в списках можно использовать локально внутри списка списке
% \setlength{\parskip}{0pt} % 
% \setlength{\parsep}{0pt}  % 

%>>>>>>>>>>>>>>>>> ТИТУЛЬНАЯ СТРАНИЦА >>>>>>>>>>>>>>>>>>>>>
\include{titlepage}
%<<<<<<<<<<<<<<<<< ТИТУЛЬНАЯ СТРАНИЦА <<<<<<<<<<<<<<<<<<<<<


%>>>>>>>>>>>>>>>>>>>>> СОДЕРЖАНИЕ >>>>>>>>>>>>>>>>>>>>>>>>>
% Содержание
\tableofcontents
%<<<<<<<<<<<<<<<<<<<<< СОДЕРЖАНИЕ <<<<<<<<<<<<<<<<<<<<<<<<<


%%%%%%%%%%%%%%%%%%%%%%% КОД РАБОТЫ %%%%%%%%%%%%%%%%%%%%%%%%
%>>>>>>>>>>>>>>>>>>>'''''''''''''''''>>>>>>>>>>>>>>>>>>>>>
\newpage
\Chapter{\lab\ \labnumber}{\labtheme}{}

\Section{Цель работы}

Изучение принципов конфигурирования и процессов функционирования 
компьютерных сетей, представляющих собой несколько подсетей, связанных с 
помощью маршрутизаторов, процессов автоматического распределения сетевых 
адресов, принципов статической маршрутизации и динамической 
маршрутизации, а также передачи данных на основе протоколов UDP и TCP.
В процессе выполнения лабораторной работы необходимо:
\begin{itemize}
    \item построить модели компьютерных сетей, представляющих собой несколько
подсетей, объединенных в одну автономную сеть, в соответствии с 
заданными вариантами топологий, представленными в Приложении  \ref{fig:application} (В1 – В6)
\item выполнить настройку сети при статической маршрутизации, 
заключающуюся в присвоении IP-адресов интерфейсам сети и ручном 
заполнении таблиц маршрутизации;
\item промоделировать работу сети при использовании динамической 
маршрутизации на основе протокола RIP и при автоматическом 
распределении IP-адресов на основе протокола DHCP;
\item выполнить тестирование построенных сетей путем проведения 
экспериментов по передаче данных на основе протоколов UDP и TCP;
\item проанализировать результаты тестирования и сформулировать выводы об 
эффективности сетей с разными топологиями;
\item сохранить разработанные модели локальных сетей для демонстрации 
процессов передачи данных при защите лабораторной работы. 
\end{itemize}

\begin{figure}[b]
    \centering
    \boxed{
    \includegraphics[width=1\linewidth]{res/application.png}
    }
    \caption{Приложение (B1 - B6)}
    \label{fig:application}
\end{figure}

\newpage
\Section{Выполнение задания}

\Subsection{Задание 1}

Сеть строится на основе сети из предыдущей лабораторной работы, но тут пришлось изменить адреса сетей, чтобы они были различными. Так же для того, чтобы компьютеры не бросали пакеты отправленные в другую сеть, нужно 
\begin{itemize}
    \item указать на них адрес шлюза (устройства в этой сети который имеет доступ в другую сеть) по умолчанию, т.е. адрес сетевой карты маршрутизатора в этой сети для <<адреса назначения>> 0.0.0.0  с маской 0.0.0.0 в меню <<таблица маршрутизации>> или из меню <<свойства>>,
    \item и включить маршрутизацию на маршрутизаторе в его меню <<свойства>>.
\end{itemize}

\begin{figure}[H]
    \centering
    \includegraphics[width=1\linewidth]{res/task-1.png}
    \caption{Э}
    \label{fig:task-1}
\end{figure}

\Subsection{Задание 2}

Сеть получается путем изменения сети из предыдущего этапа: нужно подключить 2 сеть к 3 сети через отдельный маршрутизатор. Для корректной работы сети, в каждый маршрутизатор нужно добавить статическую запись со шлюзом в сеть, в которую тот не входит. При этом каждому компьютеру в сети 2 можно будет добавить запись в таблицу маршрутизации, указывающую новый шлюз для выхода в сеть 3, иначе все пакеты сначала идут в маршрутизатор по пути в сеть 1 (шлюз по умолчанию), а потом будут пересылаться (от маршрутизатора по статической записи) в сеть 3.

\begin{figure}[H]
    \centering
    \includegraphics[width=1\linewidth]{res/task-2.png}
    \caption{Задание 2: Схема сети В2 с 2 маршрутизаторами}
    \label{fig:task-2}
\end{figure}

\Subsection{Задание 3}

Рассмотрим представленные в вариантах B3-B6 сети, учитывая при этом, что сеть 1 содержит концентратор. 

\begin{itemize}
    \item В3 обладает преимуществом простоты построения сети, ведь достаточно соединить сеть 3 и сеть 1 через коммутатор. Но при этом возникает проблема дублирования пакетов, т.к. к сети 1 подключено 2 маршрутизатора, получаемые от одного из них пакеты будут пересылаться на другой, а тот в свою очередь будет их снова посылать на концентратор. И так по кругу. К тому же на каждом компьютере в сети нужно будет прописать статическую запись в таблицу маршрутизации для шлюза не по умолчанию.
    \item B4 выглядит симметричной и тоже простой к построению, но при этом образуется сеть между маршрутизаторами. Для правильной маршрутизации, в каждый из них нужно будет добавить статическую запись для шлюза в сеть, которой он не принадлежит.
    \item В5 тоже содержит внутреннюю сеть между коммутаторами, но при этом сеть 2 и 3 еще подсоединены к двум маршрутизаторам, что заставит добавить в таблицу маршрутизации для каждого компьютер в этих сетях по статической записи для шлюза не по умолчанию.
    \item В6 выглядит как что-то среднее между В4 и В5, обладает качествами каждой из этих сетей: в сети 2 нужно будет на компьютеры добавлять статические записи в таблицу маршрутизации и между маршрутизаторами образуется сеть.
\end{itemize}

Таким образом, наиболее подходящей для построения выглядит сеть В4.

\begin{figure}[H]
    \centering
    \includegraphics[width=1\linewidth]{res/task-3.png}
    \caption{Задание 3: Схема сети В4 с 3 маршрутизаторами}
    \label{fig:task-3}
\end{figure}

\subsubsection{Настройка динамической маршрутизации по протоколу RIP}

Всем компьютерам и маршрутизаторам были добавлены программы RIP. При этом по сети начинают передаться RIP пакеты, которые конфигурируют таблицы маршрутизации в узлах сети. Но, к сожалению, по каким-то причинам не удалось добиться корректной работы сети при включении RIP: возможно нужно ждать больше времени, чтобы все пакеты передались с одного узла до другого. Либо статические записи в таблицах маршрутизации из предыдущего задания мешают корректной конфигурации узлов сети. 

\begin{figure}[H]
    \centering
    \includegraphics[width=1\linewidth]{res/task-3-rip.png}
    \caption{Задание 3: Настройка динамической маршрутизации по протоколу RIP, журналы устройств.}
    \label{fig:task-3-rip}
\end{figure}

\begin{figure}[H]
    \centering
    \includegraphics[width=1\linewidth]{res/task-3-rip-table.png}
    \caption{Задание 3: Настройка динамической маршрутизации по протоколу RIP, таблица маршрутизации.}
    \label{fig:task-3-rip-table}
\end{figure}

\subsubsection{Настройка автоматического получения сетевых настроек по протоколу DHCP}

Всем компьютерам были добавлены программы DHCP-клиентов с автоматическим управлением интерфейсами. Всем маршрутизаторам добавлены программы DHCP-серверов с динамическим назначением адресов из диапазона, содержащего их прежние (из предыдущего задания) адреса. При этом по сети начинают передаваться DHCP-пакеты запроса и предоставления адреса в соответствии с протоколом. Но, к сожалению, по каким-то причинам не удается подтвердить работоспособность сети: возможно нужно больше времени для правильной конфигурации сети, или статические записи в таблицах маршрутизации из предыдущего задания мешают корректной конфигурации.

\begin{figure}[H]
    \centering
    \includegraphics[width=\linewidth]{res/task-3-dhcp.png}
    \caption{Задание 3: Настройка автоматического получения сетевых настроек по протоколу DHCP.}
    \label{fig:task-3-dhcp}
\end{figure}

% \begin{figure}[H] % 'H' -- вставить тут же (подключен модуль), обычный вариант: 'htpb'
%     \centering
%     % { граница для иллюстрации
%     % \setlength{\fboxsep}{0pt}% убрать отсутп от границы
%     % \setlength{\fboxrule}{1pt}%
%     % \fbox{%
%             \includegraphics[width=\textwidth]{res/UML-class-diagram.png}
%     % }} % ограничение области действия параметров
%     \caption{Caption}
%     \label{fig:enter-label}
% \end{figure}

% Выполнение задания...

\Section{Вывод}

В результате выполнения работы были выполнены поставленные задачи, а именно построены сети с использованием 1, 2 и 3 маршрутизаторов, а так же для сети с 3 маршрутизаторами была настроена работа с RIP и DHCP. Во время выполнения работы были исследованы принципы построения сетей с помощью маршрутизаторов, передаваемые по сети данные, 

%<<<<<<<<<<<<<<<<<<<<<< КОД РАБОТЫ <<<<<<<<<<<<<<<<<<<<<<<<


\end{document}
%<<<<<<<<<<<<<<<< ,,,,,,,,,,,,,,,,,,,,,,, <<<<<<<<<<<<<<<<<
%<<<<<<<<<<<<<<<<<<< СОДЕРЖИМОЕ ОТЧЕТА <<<<<<<<<<<<<<<<<<<<
