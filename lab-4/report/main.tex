%%%%%%%%%%%%%%%%%%%%%%%%%%%%%%%%% LAB-5 %%%%%%%%%%%%%%%%%%%%%%%%%%%%%%%%%%
%>>>>>>>>>>>>>>>>>>>>>>>>>> ПЕРЕМЕННЫЕ >>>>>>>>>>>>>>>>>>>>>>>>>>>>>>>>>>>
%>>>>> Информация о кафедре
%\newcommand{\year}{2021 г.}  % Год устанавливается автоматически
\newcommand{\city}{Санкт-Петербург}  %  Футер, нижний колонтитул на титульном листе
\newcommand{\university}{Национальный исследовательский университет ИТМО}  % первая строка
\newcommand{\department}{Факультет программной инженерии и компьютерной техники}  % Вторая строка
\newcommand{\major}{Направление программная инженерия}  % Треьтя строка
% Пусть будет. Проще закоментить лишнее.
\newcommand{\education}{Образовательная программа системное и прикладное программное обеспечение}  % четвертая строка
\newcommand{\specialization}{Специализация системное программное обеспечение}  % пятая строка

%<<<<< Информация о кафедре

%>>>>> Назание работы
\newcommand{\reporttype}{ОТЧЕТ ПО ДОМАШНЕЙ РАБОТЕ} % тип работы, (главный заголовок титульного листа)
\newcommand{\lab}{Лабораторная работа}          % вид работы
\newcommand{\labnumber}{№ 4}                    % порядковый номер работы
\newcommand{\subject}{Компьютерные сети}         % учебный предмет
\newcommand{\labtheme}{Работа с инструментом Wireshark \\и анализ сетевого трафика}            % Тема лабораторной работы

\newcommand{\student}{Тюрин Иван Николаевич}    % определение ФИО студента
\newcommand{\studygroup}{P33102}                 % определение учебной группы 
\newcommand{\teacher}{% принимающий
    Авксентьева Е. Ю.,\\[1mm]% ФИО лектора
    Алиев Т. И.% ФИО практика
}

%>>>>>>>>>>>>>>>>>>>>>> ПРЕАМБУЛА >>>>>>>>>>>>>>>>>>>>>>>>>
\include{preamble}
%<<<<<<<<<<<<<<<<<<<<<< ПРЕАМБУЛА <<<<<<<<<<<<<<<<<<<<<<<<<



%%%%%%%%%%%%%%%%%%% СОДЕРЖИМОЕ ОТЧЕТА %%%%%%%%%%%%%%%%%%%%%
%>>>>>>>>>>>>>>> ''''''''''''''''''''''' >>>>>>>>>>>>>>>>>>
\begin{document}


%>>>>>>>>>>>>>>>> ОПРЕДЕЛЕНИЕ НАЗВАНИЙ >>>>>>>>>>>>>>>>>>>>
% Переоформление некоторых стандартных названий
%\renewcommand{\chaptername}{Лабораторная работа}
\renewcommand{\chaptername}{\lab\ \labnumber} % переименование глав
\def\contentsname{Содержание} % переименование оглавления
%<<<<<<<<<<<<<<<< ОПРЕДЕЛЕНИЕ НАЗВАНИЙ <<<<<<<<<<<<<<<<<<<<
% \setlength{\itemsep}{0pt} % установка расстояния между строчками в списках можно использовать локально внутри списка списке
% \setlength{\parskip}{0pt} % 
% \setlength{\parsep}{0pt}  % 

%>>>>>>>>>>>>>>>>> ТИТУЛЬНАЯ СТРАНИЦА >>>>>>>>>>>>>>>>>>>>>
\include{titlepage}
%<<<<<<<<<<<<<<<<< ТИТУЛЬНАЯ СТРАНИЦА <<<<<<<<<<<<<<<<<<<<<


%>>>>>>>>>>>>>>>>>>>>> СОДЕРЖАНИЕ >>>>>>>>>>>>>>>>>>>>>>>>>
% Содержание
\tableofcontents
%<<<<<<<<<<<<<<<<<<<<< СОДЕРЖАНИЕ <<<<<<<<<<<<<<<<<<<<<<<<<


%%%%%%%%%%%%%%%%%%%%%%% КОД РАБОТЫ %%%%%%%%%%%%%%%%%%%%%%%%
%>>>>>>>>>>>>>>>>>>>'''''''''''''''''>>>>>>>>>>>>>>>>>>>>>
\newpage
\Chapter{\lab\ \labnumber}{\labtheme}{}

\Section{Цель работы}

Цель работы – изучить структуру протокольных блоков данных, 
анализируя реальный трафик на компьютере студента с помощью бесплатно 
распространяемой утилиты Wireshark.

В процессе выполнения домашнего задания выполняются наблюдения за 
передаваемым трафиком с компьютера пользователя в Интернет и в обратном 
направлении. Применение специализированной утилиты Wireshark позволяет 
наблюдать структуру передаваемых кадров, пакетов и сегментов данных 
различных сетевых протоколов. При выполнении УИР рекомендуется выполнить 
анализ последовательности команд и определить назначение служебных данных, 
используемых для организации обмена данными в протоколах: ARP, DNS, FTP, 
HTTP, DHCP.

\textbf{Задание}
\begin{enumerate}
    \item Запустить Wireshark (иногда для этого требуются права 
Администратора). В появившемся окне выбрать интерфейс, для которого 
необходимо осуществлять анализ проходящих через него пакетов. В 
качестве интерфейса, используемого для захвата трафика, выбрать 
физический адаптер, через который компьютер подключён к Интернету 
(обычно этот адаптер называется Local или «Подключение по локальной 
сети»). Если меню для выбора адаптера не появляется при запуске 
Wireshark, нужно запустить из «Меню» команду «Capture->Options».
После выбора адаптера, нужно запустить процесс захвата трафика 
(кнопка Start).
\item Инициировать процесс передачи трафика по сети (например, в браузере 
открыть сайт, заданный по варианту, или запустить соответствующую 
сетевую утилиту – см. ниже).
\item Установить значение «Фильтра», чтобы из всего множества 
перехватываемых пакетов Wireshark отобразил только те, которые имеют 
отношение к выполняемому заданию. Для корректного создания фильтра 
следует пользоваться всплывающими подсказками Wireshark, которые 
активизируются при наборе фильтра. В качестве альтернативного 
способа можно использовать интерактивный конструктор фильтра, 
нажав на кнопку «Expression» в правой части элемента «Фильтр».
\item Дождаться появления данных в списке захваченных пакетов и убедиться, 
что количество пакетов достаточно для выполнения задания.
\item  Сохранить захваченный трафик в файл-трассу (pcap). Указанный файл 
нужно предъявить по первому требованию преподавателя во время 
защиты, если в этом возникнет необходимость.
\item  Описать в отчёте структуру наблюдаемых PDU (кадров, пакетов, 
сегментов) как для запросов, так и ответов. Указать название и 
назначение всех заголовков всех уровней OSI-модели в пакетах с учётом 
порядка инкапсуляции (для этого нужно раскрывать соответствующие 
значки «+» в поле с детальной информацией о выбранном пакете).
\item  Написать в отчёте ответы на вопросы задания (для этого может 
потребоваться самостоятельно изучить назначение соответствующей 
заданию сетевой утилиты, использованной для создания трафика).
\item  Поместить в отчёт скриншоты окна Wireshark, иллюстрирующие ответы 
из вышеуказанных п.6 и п.7.
\end{enumerate}

Адрес сайта, в котором по очереди встречаются инициалы (ФИО) 
студента в латинской транскрипции:

\begin{align*}
    \text{Тюрин Иван Николаевич} \to\text{tin}  \to  \text{tinkoff.ru}
\end{align*}

Найдем нужный IP-адрес с помощью команды \verb|nslookup tinkoff.ru|.

Получаем IPv4 адрес \verb|178.248.236.218|.

Для Wireshark используем маску \verb|ip.addr == 178.248.236.218|\;.

\newpage
\Section{Выполнение задания}

\Subsection{Анализ трафика утилиты ping}

Команды для выполнения: \verb|ping -l 100 tinkoff.ru|\;.

Shell с командой ping: \ref{fig:ping-shell}; результат захвата трафика в Wireshark: \ref{fig:ping-wireshark}.

\begin{figure}[h]
    \centering
    \includegraphics[width=0.8\linewidth]{res/ping-shell.png}
    \caption{Enter Caption}
    \label{fig:ping-shell}
\end{figure}

\begin{figure}[h]
    \centering
    \includegraphics[width=1\linewidth]{res/ping-wireshark.png}
    \caption{Ping в Wireshark}
    \label{fig:ping-wireshark}
\end{figure}

\begin{enumerate}
    \item Имеет ли место фрагментация исходного пакета, какое поле на это 
указывает? --- Да, при больших пакетах, появились фрагменты Fragmented IP protocol
    \item Какая информация указывает, является ли фрагмент пакета последним 
или промежуточным? --- Флаг MF = 0.
        \begin{figure}[H]
            \centering
            \includegraphics[width=0.5\linewidth]{res/ping-mf.png}
            \caption{ping MF}
            \label{fig:ping-mf}
        \end{figure}
    \item Чему равно количество фрагментов при передаче ping-пакетов? --- 3)	Количество фрагментов при передаче ping-пакетов может изменяться в зависимости от размера пакета и максимального размера передаваемых данных
    \item Построить график, в котором на оси абсцисс находится размер\_пакета, а 
по оси ординат – количество фрагментов, на которое был разделён 
каждый ping-пакет.
        \begin{figure}[H]
            \centering
            \includegraphics[width=0.75\linewidth]{res/ping-fragmentation.png}
            \caption{Фрагментация ping-пакетов}
            \label{fig:ping-fragmantation}
        \end{figure}
    \item Как изменить поле TTL с помощью утилиты ping? --- Добавить ключ -i
    \item Что содержится в поле данных ping-пакета? --- Символы английского алфавита
\end{enumerate}

\Subsection{ Анализ трафика утилиты tracert (traceroute)}

Команда для выполнения: \verb|tracert -d tinkoff.ru|\;.

Терминал с командой \verb|tracert|: \ref{fig:tracert-shell}; результат захвата трафика в Wireshark: \ref{fig:tracert-wireshark}.

\begin{figure}[h]
    \centering
    \includegraphics[width=1\linewidth]{res/tracert-shell.png}
    \caption{Tracert в консоли}
    \label{fig:tracert-shell}
\end{figure}

\begin{figure}
    \centering
    \includegraphics[width=1\linewidth]{res/tracert-wireshark.png}
    \caption{Tracert в Wireshark}
    \label{fig:tracert-wireshark}
\end{figure}

\begin{enumerate}
    \item Сколько байт содержится в заголовке IP? Сколько байт содержится в 
    поле данных? --- 20 + 64 байт
    \begin{figure}[H]
        \centering
        \includegraphics[width=1\linewidth]{res/tracert-data-size.png}
        \caption{Tracert data size}
        \label{fig:tracert-datasize}
    \end{figure}
    \item Как и почему изменяется поле TTL в следующих друг за другом ICMP пакетах tracert? Для ответа на этот вопрос нужно проследить изменение TTL при передаче по маршруту, состоящему из более чем двух хопов. --- Увеличивается на 1 каждый 3 пакет, для выявления расстояния в хопах.
    \begin{figure}[H]
        \centering
        \includegraphics[width=1\linewidth]{res/tracert-ttl.png}
        \caption{Tracert TTL}
        \label{fig:tracert-ttl}
    \end{figure}
    \item Чем отличаются ICMP-пакеты, генерируемые утилитой tracert, от ICMP пакетов, генерируемых утилитой ping (см. предыдущее задание). --- В tracert происходит увеличение TTL
    \item Чем отличаются полученные пакеты «ICMP reply» от «ICMP error» и зачем нужны оба этих типа ответов? --- Различные значения в поле TYPE. ICMP reply: получение ответного сообщения; ICMP error: ошибка. 
    \item Что изменится в работе tracert, если убрать ключ «-d»? Какой 
    дополнительный трафик при этом будет генерироваться? --- ключ -d используется для того, чтобы предотвратить преобразование IP-адресов в имена хостов, eсли убрать ключ -d при использовании tracert, то утилита будет пытаться разрешать DNS-имена для каждого узла на пути
\end{enumerate}

\Subsection{Анализ HTTP-трафика}

Результат перехвата трафика в Wireshark: \ref{fig:http-wireshark}. Как можно видеть, HTTP-трафик скрывается за TLS, так что для просмотра заголовков GET-запроса нужно отслеживать сайт без TLS или каким-то образом добавить свои <<debug>>-сертификаты в Wireshark.

\begin{figure}[h]
    \centering
    \includegraphics[width=1\linewidth]{res/http-wireshark.png}
    \caption{Http в Wireshark}
    \label{fig:http-wireshark}
\end{figure}

Для сайта без TLS результат будет примерно такой:

\begin{figure}[H]
    \centering
    \includegraphics[width=1\linewidth]{res/http-no-tls.png}
\end{figure}

\begin{figure}[H]
    \centering
    \includegraphics[width=1\linewidth]{res/http-no-tls-2.png}
\end{figure}

\Subsection{Анализ DNS-трафика}

Воспользуемся командой для сброса DNS-настроек в системе: \verb|ipconfig /flushdns|\;(рис. \ref{fig:dns-shell}).

\begin{figure}[h]
    \centering
    \includegraphics[width=1\linewidth]{res/dns-shell.png}
    \caption{Сброс настроек DNS}
    \label{fig:dns-shell}
\end{figure}

Результат перехвата трафика в Wireshrk в принципе совпадает с результатом предыдущего задания.

\begin{figure}
    \centering
    \includegraphics[width=1\linewidth]{res/dns-wireshark-no-filter.png}
    \caption{Перехват трафика DNS в Wireshark без фильтрации}
    \label{fig:dns-wireshark}
\end{figure}

\begin{enumerate}
    \item Почему адрес, на который отправлен DNS-запрос, не совпадает с 
адресом посещаемого сайта? --- Адрес отправки соответствует шлюзу по умолчанию, так как очищен кэш и нужно получить с DNS сервера адрес запрашиваемого сайта.
    \item Какие бывают типы DNS-запросов? --- \begin{itemize}
            \item \textit{Итеративный запрос} посылает доменное имя DNS-серверу и просит вернуть либо IP-адрес этого домена, либо имя DNS-сервера, авторитативного для этого домена. При этом сервер DNS не опрашивает другие серверы для получения ответа. 
            \item \textit{Рекурсивный запрос} посылает DNS-серверу доменное имя и просит возвратить IP-адрес запрошенного домена. При этом сервер может обращаться к другим DNS-серверам.
            \item \textit{Обратный запрос} посылает IP и просит вернуть доменное имя.
        \end{itemize}
    \item В какой ситуации нужно выполнять независимые DNS-запросы для 
получения содержащихся на сайте изображений? --- Когда картинки лежат на другом доменном имени.
\end{enumerate}

\Subsection{Анализ ARP-трафика}

Очистим таблицу ARP с помощью команды: \verb|netsh interface ip delete arpcache|\;, а проверим результат с помощью \verb|arp -a|\;. Результат очистки кажется неоднозначным, т.к. почти сразу там появляются новые записи, см. рис. \ref{fig:arp-shell}.

\begin{figure}[h]
    \centering
    \includegraphics[width=1\linewidth]{arp-shell.png}
    \caption{Результат очистки ARP-таблицы}
    \label{fig:arp-shell}
\end{figure}

\begin{enumerate}
    \item Какие МАС-адреса присутствуют в захваченных пакетах ARP протокола? Что означают эти адреса? Какие устройства они идентифицируют? --- хостовой компьютер и мобильную точку доступа в телефоне.
        \begin{figure}[H]
            \centering
            \includegraphics[width=1\linewidth]{res/arp-devices.png}
        \end{figure}
    \item Какие МАС-адреса присутствуют в захваченных HTTP-пакетах и что 
    означают эти адреса? Что означают эти адреса? Какие устройства они 
    идентифицируют? --- адрес отправляемого устройства, адрес принимающего устройства
    \begin{figure}[H]
        \centering
        \includegraphics[width=1\linewidth]{res/arp-http-devices.png}
    \end{figure}
    \item Для чего ARP-запрос содержит IP-адрес источника? --- Что бы добавить информацию о узле в ARP таблицу
\end{enumerate}

\Subsection{Анализ трафика утилиты nslookup}

Выполним следующие команды: \verb|nslookup tinkoff.ru| и \verb|nslookup -type=NS tinkoff.ru|\;. Результат выполнения смотри на рис. \ref{fig:nslookup-shell}.

\begin{figure}[h]
    \centering
    \includegraphics[width=1\linewidth]{res/nslookup-shell.png}
    \caption{Результат  выполнения Nslookup.}
    \label{fig:nslookup-shell}
\end{figure}

\begin{enumerate}
    \item Чем различается трасса трафика в п.2 и п.4, указанных выше? --- При запуске в п.2 утилита ищет IP-адрес хоста (запись типа A (IPv4) или AAAA (IPv6)).
    При запуске в п.4 утилита ищет Name Server для запрашиваемого хоста.
    \item Что содержится в поле «Answers» DNS-ответа? --- Данные запрашиваемого типа DNS-записи: для A - IPv4-адрес, для NS - список authoritative Name Server
    \item Каковы имена серверов, возвращающих авторитативный (authoritative) отклик? --- Авторитативный отклик возвращают серверы, которые являются ответственными 
    за зону, в которой описана информация, необходимая DNS-клиенту
\end{enumerate}

\Subsection{Анализ FTP-трафика}

Установим в Wireshrk фильтр \verb/ftp || ftp-data/\;. 

\begin{enumerate}
    \item Сколько байт данных содержится в пакете FTP-DATA? --- Размер может быть любы, но не больше MTU.
    \item Как выбирается порт транспортного уровня, который используется для 
передачи FTP-пакетов? --- Для потока управления на сервере используется порт 21. Для передачи данных используется порт 20, если передача идет в активном режиме, либо с любого 
порта клиента к любому порту сервера в пассивном режиме.
    \item Чем отличаются пакеты FTP от FTP-DATA? --- FTP используется для выполнения команд (request/response), а FTP-DATA работает с файлами.
\end{enumerate}

\Subsection{Анализ DHCP-трафика}

Установим в Wireshark фильтр \verb|bootp|. 

Выполним команды \verb|ipconfig /release| для сброса IP и \verb|ipconfig /renew| для обновления IP; результат см. на рис. \ref{fig:dhcp-shell-release} и рис. \ref{fig:dhcp-shell-renew}. Результат перехвата трафика в Wireshark можно видеть на рис. \ref{fig:dhcp-wireshark}.

\begin{figure}[h]
    \centering
    \includegraphics[width=1\linewidth]{res/dhcp-shell-release.png}
    \caption{Результат сброса IP}
    \label{fig:dhcp-shell-release}
\end{figure}

\begin{figure}[h]
    \centering
    \includegraphics[width=1\linewidth]{res/dhcp-shell-renew.png}
    \caption{Результат обновления IP}
    \label{fig:dhcp-shell-renew}
\end{figure}

\begin{figure}
    \centering
    \includegraphics[width=1\linewidth]{res/dhcp-wireshark.png}
    \caption{Результат перехвата трафика в Wireshark при работе с DHCP}
    \label{fig:dhcp-wireshark}
\end{figure}

\begin{enumerate}
    \item Чем различаются пакеты «DHCP Discover» и «DHCP Request»? --- Оба запроса выполняются клиентом, DHCP Discover ищет DHCP-сервер в своей 
канальной среде, а DHCP Request принимает предлагаемый адрес и уведомляет 
DHCP-сервер об этом.
    \item Как и почему менялись MAC- и IP-адреса источника и назначения в 
    переданных DHCP-пакетах. --- В качестве MAC-адреса источника клиент изначально подставил свой MAC-адрес, а MAC-адрес сервера он не знает, поэтому использует широковещательный MAC адрес. Соответственно, в заголовке IP-пакета в качестве адреса источника клиент 
    использовал 0.0.0.0. При отправке Offer или ACK пакетов, адреса источника 
    соответствуют адресам DHCP-сервера, адреса назначения широковещательные.
    \item Каков IP-адрес DHCP-сервера? --- 172.128.16.1
    \begin{figure}[H]
        \centering
        \includegraphics[width=1\linewidth]{res/dhcp-server-ip.png}
    \end{figure}
    \item Что произойдёт, если очистить использованный фильтр «bootp»? --- Будут отображаться все пакеты.
\end{enumerate}

\Subsection{Анализ Telegram-трафика}
\begin{itemize}
    \item Чем различаются пакета разных видов Telegram-трафика (текст, аудио, видео)? --- Текстовые данные передаются с помощью TSL. При загрузке аудио и видео используются TCP и SSLv2. Во время аудио-звонков используется TCP с SSL.
        \begin{figure}[H]
            \centering
            \includegraphics[width=1\linewidth]{res/telegram-wireshark.png}
        \end{figure}
    \item Какой Wireshark-фильтр следует использовать для независимой идентификации Telegram-трафика разных видов (текст, аудио, видео)?   
    --- В Wireshark можно установить фильтр по протоколу, соответственно нужно 
    установить \verb|tsl|, \verb|tcp|,  \verb|ssl|. При этом можно обнаружить, что весь трафик направляется на один (или несколько) IP адресов, поэтому можно еще фильтровать по нему, как я и сделал.
\end{itemize}

% \begin{figure}[H] % 'H' -- вставить тут же (подключен модуль), обычный вариант: 'htpb'
%     \centering
%     % { граница для иллюстрации
%     % \setlength{\fboxsep}{0pt}% убрать отсутп от границы
%     % \setlength{\fboxrule}{1pt}%
%     % \fbox{%
%             \includegraphics[width=\textwidth]{res/UML-class-diagram.png}
%     % }} % ограничение области действия параметров
%     \caption{Caption}
%     \label{fig:enter-label}
% \end{figure}

\Section{Вывод}

В ходе лабораторной работы был проанализирован сетевой трафик с помощью 
программы Wireshark. Изучили, какие пакеты передаются при работе утилит ping, и 
tracert и какую информацию они содержат. Также был проведен анализ трафика HTTP запросов и влияние на него кэширования данных. Кэширование также влияет на работу 
DNS, во время выполнения работы нам необходимо было очистить кэши и посмотреть на 
работу DNS-запросов. Далее был рассмотрен трафик при выполнении arp-запросов, для 
этого нужно было очистить arp-таблицу. Кроме того был проанализирован трафик при 
работе с FTP. И последним был рассмотрен трафик Telegram, мы узнали, какие протоколы 
используются в нем для передачи разных типов данных

%<<<<<<<<<<<<<<<<<<<<<< КОД РАБОТЫ <<<<<<<<<<<<<<<<<<<<<<<<


\end{document}
%<<<<<<<<<<<<<<<< ,,,,,,,,,,,,,,,,,,,,,,, <<<<<<<<<<<<<<<<<
%<<<<<<<<<<<<<<<<<<< СОДЕРЖИМОЕ ОТЧЕТА <<<<<<<<<<<<<<<<<<<<
