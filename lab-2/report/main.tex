%%%%%%%%%%%%%%%%%%%%%%%%%%%%%%%%% LAB-5 %%%%%%%%%%%%%%%%%%%%%%%%%%%%%%%%%%
%>>>>>>>>>>>>>>>>>>>>>>>>>> ПЕРЕМЕННЫЕ >>>>>>>>>>>>>>>>>>>>>>>>>>>>>>>>>>>
%>>>>> Информация о кафедре
%\newcommand{\year}{2021 г.}  % Год устанавливается автоматически
\newcommand{\city}{Санкт-Петербург}  %  Футер, нижний колонтитул на титульном листе
\newcommand{\university}{Национальный исследовательский университет ИТМО}  % первая строка
\newcommand{\department}{Факультет программной инженерии и компьютерной техники}  % Вторая строка
\newcommand{\major}{Направление программная инженерия}  % Треьтя строка
% Пусть будет. Проще закоментить лишнее.
\newcommand{\education}{Образовательная программа системное и прикладное программное обеспечение}  % четвертая строка
\newcommand{\specialization}{Специализация системное программное обеспечение}  % пятая строка

%<<<<< Информация о кафедре

%>>>>> Назание работы
\newcommand{\reporttype}{ОТЧЕТ ПО ДОМАШНЕЙ РАБОТЕ} % тип работы, (главный заголовок титульного листа)
\newcommand{\lab}{Лабораторная работа}          % вид работы
\newcommand{\labnumber}{№ 2}                    % порядковый номер работы
\newcommand{\subject}{Компьютерные сети}         % учебный предмет
\newcommand{\labtheme}{Моделирование компьютерных сетей в среде NetEmul: Локальные сети}            % Тема лабораторной работы

\newcommand{\student}{Тюрин Иван Николаевич}    % определение ФИО студента
\newcommand{\studygroup}{P33102}                 % определение учебной группы 
\newcommand{\teacher}{% принимающий
    Авксентьева Е. Ю.,\\[1mm]% ФИО лектора
     Алиев Т. И.% ФИО практика
}

%>>>>>>>>>>>>>>>>>>>>>> ПРЕАМБУЛА >>>>>>>>>>>>>>>>>>>>>>>>>
\include{preamble}
%<<<<<<<<<<<<<<<<<<<<<< ПРЕАМБУЛА <<<<<<<<<<<<<<<<<<<<<<<<<



%%%%%%%%%%%%%%%%%%% СОДЕРЖИМОЕ ОТЧЕТА %%%%%%%%%%%%%%%%%%%%%
%>>>>>>>>>>>>>>> ''''''''''''''''''''''' >>>>>>>>>>>>>>>>>>
\begin{document}


%>>>>>>>>>>>>>>>> ОПРЕДЕЛЕНИЕ НАЗВАНИЙ >>>>>>>>>>>>>>>>>>>>
% Переоформление некоторых стандартных названий
%\renewcommand{\chaptername}{Лабораторная работа}
\renewcommand{\chaptername}{\lab\ \labnumber} % переименование глав
\def\contentsname{Содержание} % переименование оглавления
%<<<<<<<<<<<<<<<< ОПРЕДЕЛЕНИЕ НАЗВАНИЙ <<<<<<<<<<<<<<<<<<<<
% \setlength{\itemsep}{0pt} % установка расстояния между строчками в списках можно использовать локально внутри списка списке
% \setlength{\parskip}{0pt} % 
% \setlength{\parsep}{0pt}  % 

%>>>>>>>>>>>>>>>>> ТИТУЛЬНАЯ СТРАНИЦА >>>>>>>>>>>>>>>>>>>>>
\include{titlepage}
%<<<<<<<<<<<<<<<<< ТИТУЛЬНАЯ СТРАНИЦА <<<<<<<<<<<<<<<<<<<<<


%>>>>>>>>>>>>>>>>>>>>> СОДЕРЖАНИЕ >>>>>>>>>>>>>>>>>>>>>>>>>
% Содержание
\tableofcontents
%<<<<<<<<<<<<<<<<<<<<< СОДЕРЖАНИЕ <<<<<<<<<<<<<<<<<<<<<<<<<


%%%%%%%%%%%%%%%%%%%%%%% КОД РАБОТЫ %%%%%%%%%%%%%%%%%%%%%%%%
%>>>>>>>>>>>>>>>>>>>'''''''''''''''''>>>>>>>>>>>>>>>>>>>>>
\newpage
\Chapter{\lab\ \labnumber}{\labtheme}{}

\Section{Цель работы}
Изучение принципов построения и настройки моделей компьютерных 
сетей в среде NetEmul.
\begin{itemize}
    \item  В процессе выполнения лабораторной работы (ЛР) необходимо:
\item построить три простейшие модели компьютерной сети;
\item выполнить настройку сети, заключающуюся в присвоении IP-адресов 
интерфейсам сети;
\item выполнить тестирование разработанных сетей путем проведения 
экспериментов по передаче данных на основе протокола UDP;
\item сохранить разработанные модели компьютерных сетей для демонстрации 
процессов передачи данных при защите лабораторной работы. 
\end{itemize}

\Section{Получение варианта}

Вариант для работы: 8 в списке группы в ИСУ университета дает 8 вариант в таблице.\\

Количество компьютеров:
\begin{enumerate}
    \item в сети 1 $N_1$ = 3
    \item в сети 1 $N_2$ = 2
    \item в сети 1 $N_3$ = 2
\end{enumerate}

Класс IP-адресов:  С

\begin{itemize}
    \item Для класса A: (Ф+Н).(И+Н).(О+Н).(Ф+И)
$$
    (5+02).(4+02).(10+02).(5+4) = 7.6.12.9
$$
    \item Для класса B: (И+Н+128).(О+Н).(Ф+Н).(Ф+И)
$$
    (4+02+128).(10+02).(5+02).(5+4) = 134.12.7.9
$$
    \item Для класса C: (192+Н +О).(Ф+Н).(И+Н).(Ф+И)
$$
    (192+02+10).(5+02).(4+02).(5+4) = 204.7.6.9
$$
\end{itemize}
Здесь: Ф, И, О – количество букв в Фамилии, Имени, Отчестве студента;
Н – две последние цифры в номере группы.

% \begin{itemize}
%     \setlength{\itemsep}{0pt} % Сокращение межстрочных расстояний
%     \setlength{\parskip}{0pt}
%     \setlength{\parsep}{0pt} 
%     \item 1
%     \item 2
% \end{itemize}


\newpage
\Section{Выполнение задания}

\Subsection{Этап 1}

Построена локальная сеть класса C с $N_1=3$ компьютеров соединенных через концентратор. Каждому компьютеру присвоен IPv4 адрес в соответствии с вариантом. После подключения компьютеров и задания их сетевым картам IP-адресов с соответствующими классу масками произошел обмен ARP пакетами и компьютеры автоматически составили таблицы маршрутизации.

Сеть была протестирована с помощью отправки UDP и TCP пакетов с устройства на устройство. Отправка TCP пакетов происходит по частям: сначала устанавливается соединение с помощью отправки пакета с флагом SYN, на которых адресат отвечает пакетом с флагом SYN, ACK; для каждой части дожидается сообщение с флагом ACK, для последней части в последнем пакете передается флаг FIN означающий конец передачи. 

При общении в этой сети все пакеты, отправляемые с одного компьютера, получают все компьютеры в сети, так, например, компьютер 2 получает пакеты, отправляемые с компьютера 3 на компьютер 1.

\begin{figure}[H]
    \centering
    \includegraphics[width=1\linewidth]{res/task-1.png}
    \caption{Этап 1: Схема сети 1 с $N_1=3$ компьютеров соединенных концентратором}
    \label{fig:task-1}
\end{figure}

\Subsection{Этап 2}

Процесс настройки сети для этого этапа аналогичен процессу настройки для  предыдущего этапа. 

Теперь при общении в сети пакеты получают только устройства участвующие в коммуникации: коммутатор и компьютеры 1 и 3, компьютер 2 не получает не предназначенных ему пакетов, но лишь в том случае, если в коммутаторе есть настроенная (в том числе динамически) таблица маршрутизации.

\begin{figure}[H]
    \centering
    \includegraphics[width=1\linewidth]{res/task-2.png}
    \caption{Этап 2: Схема сети 2 с $N_2=2$ компьютеров соединенных коммутатором (представлено 3 компьютера для демонстрации работы)}
    \label{fig:task-2}
\end{figure}

\Subsection{Этап 3}

Процесс настройки сети для этого этапа аналогичен настройке для предыдущих этапов. Коммутатор третьей сети подключается к коммутатору второй сети, т.к. у них обоих есть свободные LAN-порты (из 4 заявленных).

При передаче пакетов, в случае если отправитель не знает адреса получателя, отправитель посылает запрос на поиск нужного адреса через коммутатор/концентратор и получив ответ, посылает полезные данные, а отправитель сохраняет информацию о получателе в своет ARP-таблице. Таким образом сеть автоматически конфигурируется без дополнительных настроек

\begin{figure}[H]
    \centering
    \includegraphics[width=1\linewidth]{res/task-3.png}
    \caption{Этап 3: Схема сети 3 с $N_3=2$ компьютеров соединенных коммутатором}
    \label{fig:task-3}
\end{figure}

% \begin{figure}[H] % 'H' -- вставить тут же (подключен модуль), обычный вариант: 'htpb'
%     \centering
%     % { граница для иллюстрации
%     % \setlength{\fboxsep}{0pt}% убрать отсутп от границы
%     % \setlength{\fboxrule}{1pt}%
%     % \fbox{%
%             \includegraphics[width=\textwidth]{res/UML-class-diagram.png}
%     % }} % ограничение области действия параметров
%     \caption{Caption}
%     \label{fig:enter-label}
% \end{figure}

% Выполнение задания...

\Section{Вывод}

В результате выполнения работы были построены сегменты локальные сети 1, 2, 3 которые позже были объединены в многосегментную локальную сеть. В первом этапе работы была построена сеть с использованием концентратора, во втором этапе сеть с использованием коммутатора, в третьем этапе сети 1 и 2 были объедены в одну сеть и к ним подключена 3 сеть созданная с помощью коммутатора. При этом никаких настроек кроме указания IPv4-адресов устройств не потребовалось, каждый узел сети способен узнать адрес получателя в сети.

%<<<<<<<<<<<<<<<<<<<<<< КОД РАБОТЫ <<<<<<<<<<<<<<<<<<<<<<<<


\end{document}
%<<<<<<<<<<<<<<<< ,,,,,,,,,,,,,,,,,,,,,,, <<<<<<<<<<<<<<<<<
%<<<<<<<<<<<<<<<<<<< СОДЕРЖИМОЕ ОТЧЕТА <<<<<<<<<<<<<<<<<<<<
